%------------ Variables aleatorias discretas ---------------------
\newpage
\begin{tikzpicture}
	\node [mybox] (box){
		\begin{minipage}{.98\textwidth}
			\begin{tabular}{p{3cm} p{5cm} p{3cm} p{4cm} p{2cm} p{3cm} p{3cm} p{3cm} p{3cm} }
				{\bf Distribución} & {\bf Descripción} & {\bf Rango} & {\bf Func. probabilidad} & {\bf E[X]} & {\bf V[X]} & {\bf F.G. Momento} \\  \hline
				\makecell{Uniforme                                                                                                             \\ U(n)}   & \makecell{Sucesos uniformes} & x $\in \{{0,1,...,x_n}\}$         & $\frac{1}{n}$   & $\frac{n+1}{2}$ & $\frac{n^2-1}{12}$ & $\frac{1}{n}\sum_i{e^{t{x_i}}}$             \\ \hline
				\makecell{Bernoulli                                                                                                            \\ Be(p)}   & \makecell{Dos únicos sucesos \\ $\cdot$ suceso fracaso \\ $\cdot$ suceso éxito} &x $\in \{{0,1}\}$       &  $p^x(1-p)^{1-x}$ & p & p(1-p)  & $pe^t + (1-p)$                  \\ \hline
				\makecell{Binomial                                                                                                             \\ Bi(n,p)}  & \makecell{Representa el nº de éxitos \\ conseguidos cuando se realizan \\ n intentos del experimento de \\ Bernoulli} &  x $\in \{{0,1,...,n}\}$       & $\binom{n}{x}p^x(1-p)^{n-x}$  & np & np(1-p) & $[pe^t + (1-p)]^n$ \\ \hline
				\makecell{Geométrica                                                                                                           \\ Ge(n,p)}   &\makecell{Representa el número de \\ intentos necesarios para el \\ 1er éxito en una serie de \\ experimentos de Bernoulli \\ (SIN MEMORIA)}   & x $\in \{{1,...,\infty}\}$          & $p(1-p)^{x-1}$ & $\frac{1}{p}$  & $\frac{1-p}{p^2}$     & $\frac{np}{1-n(1-p)}$    \\ \hline
				\makecell{Binomial negativa                                                                                                    \\ BN(r,p)}  &\makecell{Representa el nº de intentos \\realizados en el experimento de \\ Bernoulli hasta obtener n éxitos}  & x $\in \{{n,...,\infty}\}$     & $\binom{x-1}{r-1}p^r(1-p)^{x-r}$ & $\frac{r(1-p)}{p}$  &  $\frac{r(1-p)}{p^2}$     & $\frac{np}{1-n(1-p)}$     \\ \hline
				\makecell{Poisson                                                                                                              \\ Po($\lambda$)} &\makecell{Nº de sucesos que se producen \\ en un determinado intervalo de \\tiempo para una frecuencia de \\ ocurrencia media\\ B(0,$\infty$)} & x $\in \{{0,...,\infty}\}$    & $e^{-\lambda}\frac{\lambda^i}{i!}$ & $\lambda$  &  $\lambda$     & $e^{\lambda(e^t - 1)}$          \\ \hline
			\end{tabular}
		\end{minipage}
	};
	\node[fancytitle, right=10pt] at (box.north west) {Variables aleatorias discretas};
\end{tikzpicture}

%------------ Variables aleatorias continuas ---------------------
\begin{tikzpicture}
	\node [mybox] (box){
		\begin{minipage}{.98\textwidth}
			\begin{tabular}{p{3cm} p{5cm} p{3cm} p{4cm} p{2cm} p{3cm} p{3cm} p{3cm} p{3cm} }
				{\bf Distribución} & {\bf Descripción} & {\bf Rango} & {\bf Func. probabilidad} & {\bf E[X]} & {\bf V[X]} & {\bf F.G. Momento} \\  \hline
				\makecell{Uniforme continua                                                                                                    \\ U(n)}   &\makecell{Su función densidad de \\ probabilidad es constante en \\ el intervalo (a,b) y nula e.o.c} & [a,b]         & $\frac{1}{b-a}$   & $\frac{a+b}{2}$ & $\frac{(b-a)^2}{12}$ & $\frac{e^{tb}-e^{ta}}{t(b-a)}$             \\ \hline
				\makecell{Gaussiana                                                                                                            \\ $N(\mu,\sigma^2)$} &   & [$-\infty,\infty$] &  $\frac{1}{\sigma\sqrt{2\pi}}e^{-\frac{1}{2}(\frac{x-\mu}{\sigma})^2}$ & $\mu$ & $\sigma^2$  & $e^{{\mu}t+\frac{\sigma^2t^2}{2}}$                  \\ \hline
				\makecell{Exponencial                                                                                                          \\ $Exp(\lambda)$}    &\makecell{Se suele utilizar para modelar \\ el tiempo entre dos sucesos \\ consecutivos que se producen de \\ forma aleatoria} & $[0,\infty]$      & $\lambda{e ^{-{\lambda}x}}$  & $\frac{1}{\lambda}$ & $\frac{1}{\lambda^2}$ & $(1-\frac{t}{\lambda})^{-1}$ \\ \hline
				\makecell{Erlang-K                                                                                                             \\ $E_k(\lambda)$}    &\makecell{Equivalemente a una suma \\ de k variables aleatorias  \\exponenciales de igual \\ parámetro} &$[0,\infty]$           & $\frac{\lambda^kx^{k-1}e^{-\lambda{x}}}{(k-1)!}$  &  $\frac{k}{\lambda}$     &  $\frac{k}{\lambda^2}$  &  $(1-\frac{t}{\lambda})^{-k}$  \\ \hline
				\makecell{Gamma                                                                                                                \\ $Gamma(\alpha,\beta)$}   & \makecell{Para valores de k enteros}& $[0,\infty]$     & $x^{\alpha-1}e^{-\frac{x}{\beta}}\frac{1}{\beta^\alpha{\Gamma(\alpha)}}$ & $\alpha\beta$  &  $\alpha\beta^2$     & $(1-\beta{t})^{-\alpha}$     \\ \hline
				\makecell{Hiperexponencial                                                                                                     \\ $HE(\vec{p},\vec{\lambda})$}   & \makecell{Es una combinación de un \\ conjunto de variables \\ aleatorias exponenciales}& $[0,\infty]$     & $\sum\limits_{i=1}^n{p_i{\lambda_i}{e^{-x\lambda_i}}}$ & $\sum\limits_{i=1}^n{\frac{p_i}{\lambda_i}}$  &  $(\sum\limits_{i=1}^n{\frac{2p_i}{\lambda_i^2}})-(\sum\limits_{i=1}^n{\frac{p_i}{\lambda_i}})^2$     &    \\ \hline
			\end{tabular}
		\end{minipage}
	};
	\node[fancytitle, right=10pt] at (box.north west) {Variables aleatorias continuas};
\end{tikzpicture}