%------------ Procesos de Markov ---------------------
\newpage
\soul{Procesos de Markov}
\begin{minipage}{.22\textwidth}
	{Son procesos estocásticos de menor orden}
	\begin{center} $P(x(t_{n+1}\le x_{n+1}|x(t_n)=x_n,x(t_{n-1})=x_{n-1}...) = P(x(t_{n+1}\le x_{n+1}|x(t_n)=x_n)$\end{center}
	\soul{El futuro solo depende del presente, no del pasado}
	Tienen un espacio discreto \\
	\begin{tikzpicture}[font=\sffamily]
		% Setup the style for the states
		\tikzset{node style/.style={state,
					minimum width=2cm,
					line width=1mm,
					fill=gray!20!white}}

		% Draw the states
		\node[node style] at (0, 3.5)     (xn_1)     {$X_{n+1}$};
		\node[node style] at (3.5, 3.5)     (x_n)     {$X_{n}$};
		\node[node style] at (0, 0) (pointpoint) {$...$};
		\node[node style] at (3.5, 0) (pointpoint2) {$...$};

		% Connect the states with arrows
		\draw[every loop,
			auto=right,
			line width=1mm,
			>=latex,
			draw=black,
			fill=black]
		(xn_1)     edge[bend right=20]            node {} (pointpoint)
		(x_n)     edge[bend right=20]            node {} (xn_1)
		(x_n)     edge[bend right=20, auto=left] node {} (pointpoint)
		(x_n)     edge[bend right=20, auto=left] node {} (pointpoint2)
		(pointpoint2) edge[bend right=20, auto=left] node {} (pointpoint);
	\end{tikzpicture} \\
	{\bf Cadenas de Markov Ergódicas} \\
	Desde cualquier estado podemos llegar a otro en un número finito de saltos de estado \\
	{\bf Cadenas de Markov en Tiempo Continuo (CTMC} \\
	No consideramos el factor {\bf discreto}. Ya en los arcos no tenemos la probabilidad de transición, sino la tasa (número de transiciones por unidad de tiempo).
	Podemos utilizar un parámetro equivalente:  {\bf tiempo de permanencia}. Distribución aleatoria continua (sin memoria) exponencial cuyo parámetro $\lambda$ es la  {\bf tasa de salida del nodo}.
\end{minipage}
\vfill\null
\columnbreak
%------------ Procesos de nacimiento y muerte ---------------------
\soul{Procesos de nacimiento y muerte}
\begin{minipage}{.22\textwidth}
	Los estados alcanzados solo son los vecinos. \\
	\begin{tikzpicture}[font=\sffamily]
		% Setup the style for the states
		\tikzset{node style/.style={state,
					minimum width=2cm,
					line width=1mm,
					fill=gray!20!white}}

		% Draw the states
		\node[node style] at (0, 3.5)     (cero)     {0};
		\node[node style] at (3.5, 3.5)     (uno)     {1};
		\node[node style] at (3.5, 0) (pointpoint) {$...$};
		\node[node style] at (0, 0) (n) {n};

		% Connect the states with arrows
		\draw[every loop,
			auto=right,
			line width=1mm,
			>=latex,
			draw=black,
			fill=black]
		(uno) edge[bend right=20, auto=left] node {$\mu_0$} (cero)
		(cero)     edge[bend right=20]            node {$\lambda_0$} (uno)
		(pointpoint) edge[bend right=20, auto=left] node {$\mu_1$} (uno)
		(uno)     edge[bend right=20]            node {$\lambda_1$} (pointpoint)
		(n) edge[bend right=20, auto=left] node {$\mu_{n-1}$} (pointpoint)
		(pointpoint)     edge[bend right=20]            node {$\lambda_{n-1}$} (n);
	\end{tikzpicture} \\
	$\lambda$: tasa de nacimiento \\
	$\mu$: tasa de muerte \\
	{\bf Características}
	\begin{itemize}[leftmargin=*]
		\item CMTC homogénea: La probabilidad solo depende de la duración y no del instante inicial.
		\item Nacimiento y muerte son independientes.
		\item La probabilidad de llegada (nacimiento) con K usuarios es: $\lambda_{k}\Delta{t}+\theta(\Delta{t})$
		\item La probabilidad de salida (muerte) con K usuarios es: $\mu_{k}\Delta{t}+\theta(\Delta{t})$
	\end{itemize}
\end{minipage}

\vfill\null
\columnbreak
%------------ Probabilidades ---------------------
\soul{Probabilidades}
\begin{minipage}{.22\textwidth}
	{\bf Probabilidad total}
	\begin{flalign}
		&\begin{aligned}\nonumber
			 & P_{k}(t + \Delta{t}) =                                                                                                                            \\
			 & \overbrace{P_{k-1}(t)[\lambda_{k-1}\Delta{t}+\theta(\Delta{t})]}^{\text{Probabilidad nacimiento estado k-1}} +                                    \\
			 & \overbrace{P_{k+1}(t)[\mu_{k+1}\Delta{t}+\theta(\Delta{t})]}^{\text{Probabilidad muerte estado k+1}} +                                            \\
			 & \overbrace{P_{k}(t)[(1-\lambda_{k}\Delta{t}) +  (1-\mu_{k}\Delta{t})+\theta(\Delta{t})]}^{\text{Probabilidad no nacimiento y no muerte estado k}}
		\end{aligned}&&
	\end{flalign}
	{\bf Caso particular K=0}
	\begin{flalign}
		&\begin{aligned}\nonumber
			 & P_{0}(t + \Delta{t}) =                                                                                          \\
			 & \overbrace{P_{1}(t)[\mu_{1}\Delta{t}+\theta(\Delta{t})]}^{\text{Probabilidad muerte estado 1}} +                \\
			 & \overbrace{P_{0}(t)[(1-\lambda_{0}\Delta{t}) + \theta(\Delta{t})]}^{\text{Probabilidad no nacimiento estado 0}}
		\end{aligned}&&
	\end{flalign}
	{\bf Aplicando el limite}
	\begin{flalign}
		&\begin{aligned}\nonumber
			 & dP_k(t) = \lambda_{k-1}P_{k-1}(t)+P_{k+1}(t)\mu_{k+1}- \\
			 & P_k(t)(\lambda_k+\mu_k)
		\end{aligned}&&
	\end{flalign}
	Son muy complicados de resolver por eso existe una solución más sencilla para los {\bf procesos de nacimiento y muerte}.
	{\bf Eliminamos la dependencia del tiempo} \\
	$0 = \lambda_{k-1}P_{k-1}+P_{k+1}\mu_{k+1}-P_k(\lambda_k+\mu_k)$ \\
	$0 = P_{1}\mu_{1}-P_0(\lambda_0)$ \\
	{\bf Probabilidades de estado}
	\begin{flalign}
		&\begin{aligned}\nonumber
			 & P_k = \frac{\lambda_{0}\lambda_{1}...\lambda_{k-1}}{\mu_{1}\mu_{2}...\mu_{k}}P_0 = \prod\limits_{i=0}^{k-1} \frac{\lambda_i}{\mu_{(i+1)}}P_0
		\end{aligned}&&
	\end{flalign}
	{\bf Probabilidades de sistema vacio}
	\begin{flalign}
		&\begin{aligned}\nonumber
			 & P_0 = \frac{1}{1+{\sum_{i=1}^{m}\prod_{j=0}^{i-1}\frac{\lambda_j}{\mu_{(j+1)}}}}
		\end{aligned}&&
	\end{flalign}
	Siendo m el número de servidores
\end{minipage}